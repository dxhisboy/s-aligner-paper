\section{Conclusion}
\label{Conclusion}

NGS data will continue to grow rapidly in the foreseeable future. Read
mapping is a critical and compute-intensive step for a variety of NGS
pipelines. While significant efforts have been devoted to optimizing
this task, it is still a major bottleneck. In this paper, we have
introduced S-Aligner---a highly scalable read mapper specifically
designed to fit the characteristics of the Sunway Taihu Light
supercomputer and its SW26010 architecture.  It scales almost linearly
with over 95\% parallelization efficiency when distributed over more
than 50,000 processes. As a result, S-Aligner can map $\approx$ 1.6 TB
of read data to the whole human genome in just a few
minutes. Moreover, when executed on a single node it can outperform
the popular all-mapper RazerS3 executed on eight Xeon E7-8860 CPUs
while achieving highly competitive mapping accuracy.

The design of S-Aligner teaches a number of lessons about properly
designing parallelization and communication patterns in order to
achieve both high performance and scalability on the new Sunway Taihu
Light supercomputer. First, both multithreading and SIMD vectorization
need to be employed to fully exploit the computational resources of
the SW26010 processor. Second, within an SP the fast LDM must be used
via DMA intrinsics in order to guarantee efficient intra-CG
communication. Third, a scalable inter-CG communication scheme must be
implemented in a way that efficiently hides the interconnection
network bottleneck. Fourth, asynchronous file-loading and data-sharing
strategies need to be implemented in order to effectively hide the
latency of the network file system. The techniques presented in this
paper could also be adapted to map similar applications exhibiting a
pipeline of hashing and verification, such as large-scale approximate
near duplicate object detection \cite{efficient} or de novo genome
assembly \cite{swap} onto the heterogeneous many-core cluster
architecture of Sunway Taihu Light.
