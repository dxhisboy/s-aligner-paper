\section{Introduction}
\label{Introduction}

{\em Next-generation sequencing} (NGS) technologies have
revolutionized the field of computational biology because of their
massive throughput at highly competitive cost. Nowadays, NGS data is
processed in almost every field of bioinformatics with important
applications including personalized cancer treatment and precision
medicine.  Modern NGS instruments such as Illumina sequencers
\cite{reviewngs} allow for the recording of billions of short DNA
fragments per day. The produced \emph{short reads} are usually only a
few hundred base pairs (bps) in length compared with complete genomes
of typical mammals covering billions of nucleotides. In order to
further analyze the set of delocalized reads, they are typically
mapped to a given reference genome by computing base-pair level
alignments to determine their original positions. This exhaustive
probing of possible alignments is a compute-heavy task and thus
justifies the need for massively parallel and efficient computation
patterns. Moreover, read alignment (or mapping) is crucial for many
DNA sequence analysis pipelines such as genotyping, the discovery of
single nucleotide polymorphism, or personal genomics.

The na\"ive computation of optimal local alignments between each read
and the reference genome is considered prohibitive because of the
demanding time complexity of the Smith-Waterman algorithm \cite{sw}
that is proportional to the product of the sequences' lengths. Thus,
methods based on a {\em seed-and-extend} approach are typically
employed to dramatically reduce the number of alignment positions by
considering only those regions in the reference genome which contain
an exact occurrence of a substring of a read. Among existing read
mappers one can find two predominant operation modes: {\em any-best}
and {\em all}. Bowtie2 \cite{bowtie2}, BWA \cite{bwa}, GEM \cite{gem},
and CUSHAW \cite{cushaw} run in {\em any-best} mode, which determines
only one or a few best alignments (also called {\em intervals}) of
each read to the reference genome. Other mappers such as RazerS3
\cite{razers3}, Hobbes2 \cite{hobbes2}, BitMapper \cite{bitmapper},
and mrFAST \cite{mrfast} operate in {\em all} mode, which reports all
intervals satisfying a certain constraint, for example, compliance
with a predefined edit distance threshold.  Because of their extensive
nature all-mappers demand far more computational resources than their
any-best counterparts. Nevertheless, when studying gene innovation or
phenotypic variation the complete list of alignments is crucial for a
thorough analysis.

Consequently, we pursue the design of an all-mapper in order to
support a wide range of biomedical applications. Associated runtimes
can be dramatically reduced if we target clusters with a large number
of compute nodes. In the recent past, the prevalent trend in
supercomputing has favored heterogeneous nodes featuring attached
coprocessors, such as GPUs or Xeon Phi architectures, because of their
vast compute capabilities and low energy footprint. A similar
architecture has recently been introduced with the \emph{Sunway Taihu
  Light} supercomputer, which is based on the neo-heterogeneous
SW26010 many-core processor. This cluster consists of 40,960
nodes. Each node provides 260 compute cores organized in four {\em
  compute groups} (CGs). A total of 10,649,600 cores provides Linpack
performance of up to 93 PFlop/s, rendering it the world's fastest
supercomputer in the Top500 list of November 2016.

In this paper, we introduce S-Aligner---a highly scalable read mapper
targeting the Sunway Taihu Light supercomputer and its SW26010
architecture. To the best of our knowledge, we are the first to
investigate the applicability and scalability of read mapping on this
architecture. Our approach relies on four major contributions. First,
we provide an efficient implementation of Myers' bit-parallel string
matching algorithm that harnesses the full parallelization and
vectorization potential of the SW26010 many-core processor. Our new
implementation is three orders of magnitude faster than a na\"ive
single-threaded Myers' version. Second, we exploit fast local device
memory via DMA intrinsics for intra-CG communication, which results in
a 22$\times$ speedup compared with the non-DMA variant. Third, a
highly scalable inter-CG communication scheme is proposed. This
results in a parallelization efficiency of over 95\% when using 53,248
CGs. Fourth, we employ asynchronous file loading and data-sharing
strategies to effectively hide the latency of the network file
system. We further expect that the presented techniques can guide the
design and implementation of similar types of applications on the
neo-heterogeneous many-core cluster architecture of Sunway Taihu
Light.

The rest of this paper is organized as follows: Section~\ref{Related
  Work} describes required background about the Sunway Taihu Light
architecture, related work, and the seed-and-extend approach to read
mapping. Details of the design and implementation of S-Aligner are
presented in Section~\ref{Implementation}. Performance evaluations in
terms of speed, scalability, and alignment quality are carried out in
Section~\ref{Evaluation}. Section~\ref{Conclusion} summarizes our
conclusions.
